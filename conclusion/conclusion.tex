%!TEX root = ../thesis.tex

本論文では, 事前に収集した画像と行動を用いて, end-to-end学習により経路追従行動をオフラインで模倣学習する手法に関して検討した. 目標経路上及び±0.2[m]の位置, ロボットをヨー方向に0[deg]及び±0.5[deg]回転させた際の視覚情報と目標角速度があれば, 経路追従を継続できることを確認した. 実環境を想定した実験では, 切り抜き画像とオフセットを加えた目標角速度を用いて学習する手法を検討した. 結果として, 経路追従を継続できないことを確認した. そこで, 画像と目標に問題があるかを教師データを入れ替えた実験と, カメラを9個にした実験で調査した. 調査の結果, 画像と目標角速度の両方に問題があることを明らかにした. 