%!TEX root = ../thesis.tex

\section{実験概要}
前章では, オフライン手法によりend-to-end学習器を用いて, 視覚による経路追従行動を模倣できることを確認した. 前章の実験はシミュレータ上での検証であったが, これを実ロボットに適用する場合, まず画角の広いカメラを3つ用意する. 次に, カメラをロボットの中央及び±0.2[m]に配置して, 経路に沿って走行する. 方向に関しては, 画角の広いカメラで得られた画像から, 目標経路の方向及び±5[deg]回転させた時に得られる画像を切り抜いて, データセットに加える. 各画像に対する目標角速度は, 中央のカメラ画像とペアになる角速度にオフセットを加えて得ることができる. これにより, オンライン手法では目標経路を複数回周回しなければならなかったものが, 1周すればよいことになる. 
\par そこで, この手法が有効であるかをシミュレータ上の実験により検証することを本実験の目的とする. 

\section{提案する手法}